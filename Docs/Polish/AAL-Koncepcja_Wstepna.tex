% !TEX TS-program = pdflatex
% !TEX encoding = UTF-8 Unicode

\documentclass[11pt]{article} % use larger type; default would be 10pt

\usepackage[utf8]{inputenc} % set input encoding (not needed with XeLaTeX)
\usepackage[T1]{fontenc} %font encoding ? --> need to clarify that

%%% PAGE DIMENSIONS
\usepackage{geometry} % to change the page dimensions
\geometry{a4paper} % or letterpaper (US) or a5paper or....
\geometry{margin=1in} % for example, change the margins to 2 inches all round

\usepackage{graphicx} % support the \includegraphics command and options
\usepackage[parfill]{parskip} % Activate to begin paragraphs with an empty line rather than an indent

%%% PACKAGES
\usepackage{booktabs} % for much better looking tables
\usepackage{array} % for better arrays (eg matrices) in maths
\usepackage{paralist} % very flexible & customisable lists (eg. enumerate/itemize, etc.)
\usepackage{verbatim} % adds environment for commenting out blocks of text & for better verbatim
\usepackage{subfig} % make it possible to include more than one captioned figure/table in a single float
% These packages are all incorporated in the memoir class to one degree or another...

%%% HEADERS & FOOTERS
\usepackage{fancyhdr} % This should be set AFTER setting up the page geometry
\pagestyle{fancy} % options: empty , plain , fancy
\renewcommand{\headrulewidth}{0pt} % customise the layout...
\lhead{}\chead{}\rhead{}
\lfoot{}\cfoot{\thepage}\rfoot{}
%%% END Article customizations

%%% The "real" document content comes below...

\title{Analiza Algorytmów - Projekt}
\author{Stawczyk Przemysław 293153}
\date{} % Activate to display a given date or no date (if empty),
         % otherwise the current date is printed 

\begin{document}
\maketitle

\section{Opis Zagadnienia}
\subsection{Zadane polecenie}
\paragraph{}
    Detektyw Karczoch poszukuje nieuczciwych kont na instagramie. Do tego celu analizuje komentarze dotyczące zamieszczanych zdjęć — wie, że takie konta często walczą między sobą, kupując zwolenników i malkontentów. W komentarzach do takich kont, ludzie z jednej grupy piszą tylko do ludzi z drugiej i odwrotnie. Detektyw Karczoch wie, że zwolennik nigdy nie rozmawia z innym zwolennikiem, a malkontent z malkontentem - to dla nich strata czasu.
\paragraph{}    
	Zadanie : \textsl{Przygotować program, który na podstawie zamieszczonych komentarzy oceni, czy dane konto jest nieuczciwe.}

\subsection{Przykładowe dane wejściowe}

\paragraph{Uczciwe :}
\begin{texttt}\\
  \#Jan: Piękne zdjęcie.\\
  \#Ola: @Jan Masz rację.\\
  \#Ania: @Ola Nie ma racji!\\
  \#Jan: @Ania właśnie, że mam!\\
  \#Ola: @Ania sama nie masz!\\
\end{texttt}
\paragraph{Nieuczciwe : }
\begin{texttt}\\
   \#Jan: Piękne zdjęcie.\\
   \#Ola: @Jan Nieprawda.\\
   \#Jan: @Ola czemu tak twierdzisz?\\
   \#Ania: @Ola no właśnie?\\
   \#Tomek: @Ania twierdzi tak, bo ma rację!!!\\
\end{texttt}
\subsection{Założenia}
\begin{itemize}
\item
Zakładam, że treść komentarza nie ma znaczenia więc dla opisanego celu liczy się jedynie adresat komentarza.
\item
Zakładam, że pojedyńczy komentarz może nie mieć adresata lub mieć dokładnie jednego adresata.
\item
Zakładam, że istnienie pojedyńczych komentarzy od osób które nie adresują do nikogo komentarzy ani nie są adresatami żadnego komentarza nie ma znaczenia dla rozstrzygnięcia uczciwości konta
\end{itemize}
\section{Opis Algorytmu}
\subsection{Uzywane pojęcia}
Przy opisie algorytmu stosuje następujące odwzorowanie:
\begin{itemize}
\item
Każdy komentujący jest reprezentowany przez wierzchołek grafu.
\item
Komentarz adresowany do innego użytkownika jest reprezentowany przez krawędź w grafie.
\item
Wiele komentarzy pomiędzy dwoma użytkownikami nie jest odwzorowane jako wielokrotne krawędzie lub krawędzie ważone.\\
 \textsl{Istotny jest fakt komunikacji 2 komentujących, a nie jej objęctość}
\item
Komentarz jest odwzorowany jako krawędz nieskierowana. \\
\textsl{Istotny jest fakt komunikacji 2 komentujących, a nie który z urzytkowników napisał do którego}
\end{itemize}
\subsection{Zarys działania}
\begin{paragraph}{Opis :\\}
   Proponowany algorytm opiera się na założeniu, że istnienie 2 grup \textsl{[zwolenników i malkontentów]}, które nie piszą komentarzy wewnątrz swoich grup jest równoznaczne z możliwością podzielenia wierzchołków grafu na 2 cześci w których nie ma krawędzi \textsl{[dwudzielnosc grafu]}. Pojedyńcze wierzchołki\textsl{[uzrytkownicy którzy nie piszą do nikogo, ani nie są adresatami wiadomości]} nie mają znaczenia dla rozstrzygnięcia dwudzielności grafu.
\end{paragraph}
\paragraph{Szkic kroków : }
\renewcommand{\labelenumi}{\arabic{enumi}}
\begin{enumerate}
\item
Wczytanie listy komentarzy ze standardowego wejścia lub pliku tworząc graf identyfikowany nazwami urzytkowników
\item
Usunięcie wierzchołków nieposiadających krawędzi.
\item
Przejście po wierzchołkach "kolorując" je na przeciwne kolory zaczynając od dowolnego niepokolorowanego. Następne wierzchołki brane z kolejki.
\renewcommand{\labelenumii}{\Roman{enumii}}
\begin{enumerate}
\item
Koloruj wierzchołki sąsiednie na przeciwny kolor. Umieść niepokolorowane wcześniej wierzchołki w kolejce.
\item
Gdy wierzchołek po drugiej stronie krawędzi jest w tym samym kolorze: \\STOP --> \textsl{Konto jest uczciwe.}
\end{enumerate}
\item
Usuń pokolorowane wierzchołki.
\item
Jeśli graf niepusty: idź do 3 --> \textsl{Przetwarzanie kolejnego spójnego podgrafu},\\ jeśli pusty: KONIEC --> \textsl{Konto nie jest uczciwe}
\end{enumerate}
\section{Złożoność Alorytmu}
\subsection{Analiza Złożoności Algorytmu}
Algorytm iteruje po wszystkich wierzchołkach, a w ramach każdego z wierzchołków po wszystkich jego krawędziach.\\
Z tego wynika, że złożoność pesymistyczna algorytmu powinna być klasy \textbf{O(V+E)}, gdzie \textbf{V} to ilość wierzchołków, a \textbf{E} ilość krawędzi w grafie.
\subsection{Metoda pomiarów}
Algorytm w przypadku konta uczciwego może zakończyć się wcześniej, więc aby sprawdzać złożoność algorytmu należy mierzyć czasy wykonania analizy nieuczciwych kont. Aby profilować sam algorytm, mierzony powinien być jedynie czas spędzony na obliczeniach więc będzie on mierzony dopiero po wczytaniu całego grafu z pliku do pamięci. \\\\
Dla wybranych wielkości grafu mierzony byłby czas dla kilku danych testowych. Analiza byłaby wykonana osobno ze względu na ilość krawędzi \textsl{[przy stałej ilości wierzchołków]}, oraz na ilość wierzchołków \textsl{[przy stałej ilości krawędzi, jednak większej od liczby wierzchołków]}.
\section{Implementacja}
\subsection{Technologie planowane do wykorzystania}
\begin{itemize}
\item
\textsl{C++} jako język do implementacji struktury grafu i algorytmu.
\item
\textsl{Python} lub \textsl{C++} jako zewnętrzny program do generowania plików z komentarzami.
\item
\textsl{boost::chrono} w celu pomiaru czasu spędzonego przez algorytm na procesorze. 
\end{itemize}
\subsection{Struktury danych}
\begin{itemize}
\item
"Node" - \textsl{jako klasa zawierająca nazwe użytkownika, pole określające kolor, identfikowana przez numer pozycji w dynamicznej tablicy}
\begin{itemize}[\textsl{W Node} :]
\item
vector<int> - \textsl{przechowuje wierzchołki sąsiednie.}
\item
enum "Kolor" - \textsl{określa czy wierzchołek był odwiedzony oraz kolor odwiedzoneo wierzchołka.}
\end{itemize}
\item
map<String, int> - \textsl{zapewniająca odwzorowanie nazwy użytkownika na pozycję w tablicy} \\ 
\textsl{Pomocniczo przy wczytywaniu grafu}
\item
vector<Node> - \textsl{kontener przechowujący wierzchołki}\\ 
\textsl{Zapewnia dostęp do wierzchołka poprzez numer} 
\item list<int> lub queue<int> - \textsl{przechowuje listę wierzchołków do odwiedzenia, umieszczane są w niej kolejni nieodwiedzeni sąsiedzi przetwarzanego wierzchołka}
\item
"Graph" - \textsl{wszystkie struktury danych w ramach makroklasy} 
\end{itemize}
\section{Generowanie danych wejściowych}
\subsection{Parametry:}
\begin{itemize}
\item
liczebność wierzchołków w grafie lub liczebność obu grup z osobna
\item
liczebność krawędzi w grafie
\item
uczciwość konta
\item
nazwa pliku wyjściowego
\end{itemize}
\subsection{Zarys działania:}
  Przy generowaniu danych w pierwszym kroku tworzone są 2 grupy wierzchołków o podanych licznościach lub wedle losowego podziału jednej liczby na wejściu. Następnie tworzone są krawędzie \textsl{[komentarze]} poprzez losowanie po jednym wierzchołku z każdej grupy. W przypadku zadania parametrem uczciwości konta dodawane są w małej liczbie krawędzie w obrębie grup stworzonych wcześniej. Jako ostatni etap występuje zapis danych do pliku gdzie dla każdej krawędzi losowany jest jeden ze zdefiniowanych w programi komentarzy \textsl{[dla uwiarygodnienia danych wyjściowych]}.\\
  Struktury danych analogiczne do wykorzystywanych w algorytmie.
\end{document}
